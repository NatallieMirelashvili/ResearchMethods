\documentclass[12pt]{article}

% ====== Encoding and Fonts ======
\usepackage[utf8]{inputenc}
\usepackage[T1]{fontenc}
\usepackage{lmodern}

% ====== Layout ======
\usepackage[a4paper, margin=2.5cm]{geometry}
\usepackage{setspace}
\onehalfspacing

% ====== Other Useful Packages ======
\usepackage{graphicx}
\usepackage{amsmath}
\usepackage{enumitem}

% ====== Title Information ======
\title{Project Proposal for Research Methods Course\\Spring Semester, 2025\\Ben-Gurion University of the Negev}

\begin{document}

\maketitle

\section*{1. Proposal Title}
Detection of Facial Images Generated by Artificial Intelligence: Distinguishing Between Real and Synthetic Images Using Machine Learning.
\section*{2. Students’ Information}
Natallie Mirelashvili 322685314 miralshv@post.bgu.ac.il\\
Ravit Cohen 313188799 cravi@post.bgu.ac.il\\
Omer Onn 318910759 omeron@post.bgu.ac.il\\
Yuval Aviv 315106948 avivyuv@post.bgu.ac.il\\
Yehonatan Segal 209359801 segaldol@post.bgu.ac.il


\section*{3. General Description of the Problem}
The aim of the research is to examine whether it is possible to reliably identify a difference between real facial images and images generated by artificial intelligence.

The central research hypothesis is that although AI-generated images are becoming increasingly realistic, there are still identifiable visual and statistical features that will allow computational models to distinguish them from real images.

The research questions include:
\begin{itemize}[noitemsep]
    \item Which features best distinguish between real and AI-generated faces?
    \item Which models or classification methods best fit the problem?
    \item How do human identification performances compare to those of the model?
\end{itemize}

\section*{4. Required Resources}
\begin{itemize}[noitemsep]
    \item Image Databases: Collections containing both AI-generated and real facial images.
    \item Development Environment: Visual Studio Code (VS Code) for model implementation and experimentation.
    \item Human Participants: For comparison between human perception and AI model performance.
\end{itemize}

\section*{5. General Design of the Experiments}
\begin{itemize}[noitemsep]
    \item Data collection and cleaning - using a balanced dataset of real and AI-generated faces.
    \item Model training - training multiple models on the dataset.
    \item Model evaluation - selecting appropriate metrics for performance evaluation.
    \item Model comparison - examining and comparing the results.
    \item Result interpretation - using tools to identify the features that contributed to the classification.
    \item Conclusion drawing - answering the research questions based on the findings.
\end{itemize}

\section*{6. General Description of the User Study}
As part of the evaluation stage, a human perception experiment will be conducted in which participants will be asked to distinguish between real and AI-generated images.

The participants' results will be compared to the model's results in order to examine the gap between human perception and computational recognition.

\end{document}
