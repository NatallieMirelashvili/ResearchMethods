\documentclass[12pt]{article}

% ====== Encoding and Fonts ======
\usepackage[utf8]{inputenc}
\usepackage[T1]{fontenc}
\usepackage{lmodern}

% ====== Layout ======
\usepackage[a4paper, margin=2.5cm]{geometry}
\usepackage{setspace}
\onehalfspacing

% ====== Other Useful Packages ======
\usepackage{graphicx}
\usepackage{amsmath}
\usepackage{enumitem}

% ====== Title Information ======
\title{Project Proposal for Research Methods Course\\Spring Semester, 2025\\Ben-Gurion University of the Negev}

\begin{document}

\maketitle

\section*{1. Proposal Title}
Detection of Facial Images Generated by Artificial Intelligence: Distinguishing Between Real and Synthetic Images Using Machine Learning.
\section*{2. Students’ Information}
Natallie Mirelashvili 322685314 miralshv@post.bgu.ac.il\\
Ravit Cohen 313188799 cravi@post.bgu.ac.il\\
Omer Onn 318910759 omeron@post.bgu.ac.il\\
Yuval Aviv 315106948 avivyuv@post.bgu.ac.il\\
Yehonatan Segal 209359801 segaldol@post.bgu.ac.il


\section*{3. General Description of the Problem}
The aim of the research is to examine whether it is possible to reliably identify a difference between real facial images and images generated by artificial intelligence.

The central research hypothesis is that although synthetic images are becoming increasingly realistic, there are still identifiable visual and statistical features that will allow computational models to distinguish them from real images.

The research questions include:
\begin{itemize}[noitemsep]
    \item Which visual features best distinguish between real and synthetic faces?
    \item Which models or classification methods best fit the problem?
    \item How do human identification performances compare to those of the model?
\end{itemize}

\section*{4. Required Resources (Data and Code)}
\begin{itemize}[noitemsep]
    \item Databases containing AI-generated and real images.
    \item VS-Code.
    \item Human resources for survey evaluation.
\end{itemize}

\section*{5. General Design of the Experiments}
\begin{itemize}[noitemsep]
    \item Data collection and cleaning - using a balanced dataset of real and synthetic faces.
    \item Model training - training different models on the dataset.
    \item Model evaluation - selecting suitable metrics for performing evaluation.
    \item Comparison between models - examining the results and comparing them.
    \item Explanation of results - using tools to identify the features that contributed to the classification.
    \item Drawing conclusions - answering the research questions in light of the findings.
\end{itemize}

\section*{6. General Description of the User Study}
As part of the evaluation stage, a human perception experiment will be conducted in which participants will be asked to distinguish between real and synthetic images.

The participants' results will be compared to the model's results in order to examine the gap between human perception and computational recognition.

\end{document}
